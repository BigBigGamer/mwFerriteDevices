\input{text/preamble}
\newcommand{\vH}{\textbf{H}}
\newcommand{\vE}{\textbf{E}}
\newcommand{\vB}{\textbf{B}}
\newcommand{\vD}{\textbf{D}}
\newcommand{\vr}{\textbf{r}}
\newcommand{\vj}{\textbf{j}}
\newcommand{\vk}{\textbf{k}}
\newcommand{\vx}{\textbf{x}}
\newcommand{\vy}{\textbf{y}}
\newcommand{\vz}{\textbf{z}}
\begin{document}

\def\labauthors{Карусевич А.А., Шиков А.П.}
\def\labgroup{440}
\def\labnumber{1}
\def\labtheme{Волноводные ферритовые устройства СВЧ диапазона}
\input{text/titlepage}

\newpage

{\bfseries Цель работы:} 
Изучение электродинамических систем, содержащих гиротропные элементы - ферриты.

\section{Теоритическая часть}
\subsection{Введение}
Известно, что в линейных анизотропных средах с симметричными тензорами диэлектрической и магнитной проницаемости
$\epsilon_{ij} = \epsilon_{ji},\mu_{ij} = \mu_{ji}$ (а также в обычных изотропных средах) имеет место теорема взаимности. Эта теорема оказывается несправедливой в средах
с несимметричными тензорами проницаемостей, в частности, в так называемых гиротропных средах, к которым принадлежат
плазма и ферриты, находящиеся во внешнем постоянном магнитном ноле.

Используя невзаимные свойства гиротропных сред, можно создавать устройства, канализирующие электромагнитные волны в
одном направлении и почти не пропускающие в противоположном направлении. Этим объясняется широкое применение ферритов в
волноводных устройствах сверхвысоко-частотного (СВЧ) диапазона. К настоящему времени разработано большое количество
устройств с ферритовыми элементами, имеющих различные конструктивные исполнения и электродинамические характеристики.

\subsection{Анизотропные и гиротропные среды}

Анизотропными средами называются среды, локальные макроскопические свойства которых различны в различных направлениях.
Анизотропией обладают многие жидкости и большинство кристаллов. Поликристалличе-ские вещества становятся анизотропными
под воздействием давления, статических электрических и магнитных полей. Наряду с анизотропными существуют среды,
локальные макроскопические свойства которых поиштрнапт-ны относительно зеркальных отражений, т.е. изменяются при
некоторых зеркальных отражениях. Такие среды называются оетостноппонктшшыми (пли гиротропными). Примерами таких сред
могут служить кристаллы без центра симметрии или среды, состоящие из частиц, не обладающих центром симметрии. Некоторые
среды приобретают гиротропные свойства при наложении внешнего магнитного поля. Такие среды принято называть
магнито-активными. Типичными примерами магиитоактивиых сред являются плазма и ферриты, находящиеся во внешнем постоянном
магнитном поле.

\textit{Ферриты}, представляют собой химические соединения оксида железа с оксидами других, так называемых характеризующих
металлов (никеля, кобальта, магния, марганца, кадмия и т. д.). Особенностью этих материалов является сочетание
гиротропных свойств с низкой электропроводностью, благодаря чему электромагнитные волны при определенных условиях могут
распространяться в ферритах с достаточно малым затуханием.

Магнитные свойства ферритов определяются наличием в их кристаллической решетке атомов или ионов, обладающих отличным от
нуля магнитным моментом. В отсутствие внешнего магнитного поля магнитные моменты ориентированы хаотически, и в целом
феррит изотропен. В достаточно сильном постоянном магнитном поле $\vH_0$ магнитные моменты всех атомов за время
$\tau_0 \simeq (10^{-9} - 10^{-7})$ с (время релаксации) устанавливаются по направлению магнитного поля. В этом состоянии феррит обладает
значительной намагниченностью и становится анизотропной гиротроппой средой по отношению к высокочастотному электромагнитному полю.

В линейном анизотропном магнетике каждая компонента вектора магнитной индукции $\vB$ представляет собой линейную функцию
трех компонент ноля $\vH$:
\begin{equation}
    \begin{aligned} 
        B_{x} &=\mu_{x x} H_{x}+\mu_{x y} H_{y}+\mu_{x z} H_{z} \\
        B_{y} &=\mu_{y x} H_{x}+\mu_{y y} H_{y}+\mu_{y z} H_{z} \\
        B_{z} &=\mu_{z x} H_{x}+\mu_{z y} H_{y}+\mu_{z z} H_{z}
    \end{aligned}
    \label{eq:1}
\end{equation}
Величины $\mu_{i j}(i,j=x,y,z)$, входящие в соотношения \eqref{eq:1}, являются компонентами тензора второго ранга, называемого
тензором магнитной проницаемости:
\begin{equation}
    \hat{\mu}=\left(\begin{array}
        {ccc}{\mu_{x x}} & {\mu_{x y}} & {\mu_{x z}} \\
        {\mu_{y x}} & {\mu_{y y}} & {\mu_{y z}} \\
        {\mu_{z x}} & {\mu_{z y}} & {\mu_{z z}}
    \end{array}\right)
    \label{eq:2}
\end{equation}


С учетом \eqref{eq:2} материальные уравнения \eqref{eq:1} можно записать в удобной матричной форме

\begin{equation}
    \textbf{B} = \hat{\mu}\textbf{H}
    \label{eq:3}
\end{equation}

При наличии в феррите электромагнитного поля, изменяющегося во времени но гармоническому закону и имеющего напряженность
$\vH(\vr)e^{i\omega t}$, малую но сравнению с постоянным магнитным полем $\textbf{H}_0 =
H_0\textbf{z}_0(|\vH|\ll H_0)$, тензор магнитной проницаемости $\hat{\mu}$ принимает вид

\begin{equation}
    \hat{\mu}=\left(\begin{array}
        {ccc}{\mu} & {i\mu_{a}} & {0} \\
        {-i\mu_{a}} & {\mu} & {0} \\
        {0} & {0} & {\mu_{||}}
    \end{array}\right)
    \label{eq:4}
\end{equation}

\begin{equation}
    \mu=1+\frac{4 \pi \chi \omega_{H}^{2}}{\omega_{H}^{2}-\omega^{2}}, 
    \quad \mu_{a}=\frac{4 \pi \chi \omega_{H} \omega}{\omega_{H}^{2}-\omega^{2}}, 
    \quad \mu_{ \|}=1
    \label{eq:5}
\end{equation}
Здесь $\omega$ - круговая частота, $\chi$ - статическая магнитная восприимчивость феррита, $\omega_H = \frac{e H_0}{m
c}$ - гирочастота электрона ($m$ и $e$ — масса и абсолютное значение заряда электрона соответственно, $c$ — скорость света в вакууме).


Обратим внимание на то, что тензор \eqref{eq:4} подчиняется соотношению $\mu_{ij} = \mu_{ji}^*$ т. е. является эрмитовым (знак «*» означает
операцию комплексного сопряжения). Данное свойство имеет место для сред при пренебрежении потерями.

В отсутствие статического магнитного поля $(\textbf{H}_0 = 0)$ недиагональные элементы тензора \eqref{eq:4} обращаются в пуль, диагональные
компоненты $\mu$ и $\mu_{||}$ принимают одинаковые значения ($\mu=\mu_{||}$), и феррит становится изотропным.

При стремлении частоты поля $\omega$ к гирочастоте $\omega_H$ компоненты $\mu$ и $\mu_a$ обращаются в бесконечность. Это явление называют
ферромагнитным резонансом. В действительности из-за диссипативных процессов элементы тензора $\mu$ и $\mu_a$ становятся весьма
большими, но не бесконечными. Потери в феррите в случае малой диссипации можно учесть феноменологически , заменив
частоту $\omega_H$ в выражениях \eqref{eq:5} па комплексную величину $\tilde{\omega}_H = \omega_H+ i \gamma \omega$, где
$\gamma$ — безразмерный параметр диссипации ($\gamma \ll 1$).

\begin{figure}[h!]
    \centering
    \includegraphics[width = 0.5\linewidth]{imgs/temp/001.png}
    \caption{Качественный вид зависимостей величин $\mu^{\prime}$, $\mu^{\prime \prime}(a)$ и 
    $\mu_{a}^{\prime}$, $\mu_{a}^{\prime \prime}$(b) от частоты $\omega$}
    \label{fig:1}
\end{figure}

Таким образом, при учете потерь величины $\mu$ и $\mu_{a}$ являются комплексными, т. е. $\mu=\mu^{\prime}-i \mu^{\prime \prime}, 
\mu_{a}=\mu_{a}^{\prime}-i \mu_{a}^{\prime \prime}$, а тензор $\hat{\mu}$ перестает быть эрмитовым ($\mu_{ij} \neq
\mu_{ji}^*$). Качественные зависимости действительных и мнимых частей этих величин от частоты
$\omega$ при фиксированном статическом поле $H_0$ показаны на рис. \ref{fig:1}. Из рис. \ref{fig:1} видно, что в случае $\omega=\omega_H$ и
величины $\mu^{\prime \prime}$ и $\mu_{a}^{\prime \prime}$ максимальны, что свидетельствует о резонансном поглощении
электромагнитного поля ферритом. Следует отметить, что в ограниченных ферритовых образцах внутреннее статическое
магнитное поле $\vH_0$ отличается от внешнего поля $\vH_{\text{ВН}}$, в которое
помещен образец. Поэтому частота ферромагнитного резонанса при заданном поле $\vH_{\text{ВН}}$ зависит от формы образца и его
ориентации относительно внешнего магнитного поля. Для определения внутреннего ноля в образце следует решить
соответствующую краевую задачу магнитостатики.

\subsection{Распространение электромагнитных волн в гиротропной среде}
Рассмотрим неограниченную гиротропную среду с тензором магнитной проницаемости \eqref{eq:4} и скалярной диэлектрической
проницаемостью $\epsilon$. Такую среду принято называть гиромагнитной. Уравнения Максвелла для комплексных амплитуд
$\vE(\vr)$, $\vH(\vr)$ в гиромагнитной среде без источников ($\vj^e = \vj^m=0$) имеют вид
\begin{equation}
    \rot \vE=-i k_{0} \hat{\mu} \vH, \quad \rot \vH=i k_{0} \epsilon \vE
    \label{eq:2:1}
\end{equation}
где $k_0 = \omega /c$ — волновое число в свободном пространстве. Гиротропная среда является средой двоякопреломляющей, т. е. в
ней вдоль каждого направления могут распространяться две волны с различными показателями преломления $n_{1,2}$ и фазовыми
скоростями $v_{\text{ф}1,2} = c/n_{1,2}$, являющиеся линейно независимыми решениями уравнений \eqref{eq:2:1}. Показатели преломления волн как
типа 1, так и типа 2 зависят от угла $\theta$ между волновым вектором $\vk$ соответствующей волны и постоянным магнитным полем $\vH_0$.
Ниже мы остановимся более подробно на продольном ($\theta=0$) и поперечном ($\theta=\pi/2$) распространении электромагнитных волн в
гиромагнитной среде (феррите).

Следует заметить, что, в отличие от гиротропной среды, обычная изотропная среда представляет собой некоторый вырожденный
случай, так как в ней вдоль каждого направления могут независимо распространяться две линейно поляризованные волны с
различными направлениями векторов поля, но одинаковыми фазовыми скоростями (поляризационное вырождение).
Сложение полей этих двух воли при соответствующем выборе амплитуд и сдвига фаз между ними позволяет получать в
изотропной среде волны с различной поляризацией (линейной, круговой, эллиптической).
В анизотропных и гиротроппых средах поляризационное вырождение отсутствует, и каждой волне с определенным значением
показателя преломления соответствует фиксированная поляризация, отличающаяся в общем случае от линейной.

\subsubsection{Продольное распространение электромагнитных волн в гиромагнитной среде. Эффект Фарадея}
Запишем уравнения \eqref{eq:2:1} в декартовых координатах:
\begin{equation}
    \begin{aligned}
        &\frac{\partial H_{z}}{\partial y}-\frac{\partial H_{y}}{\partial z}=i k_{0} \epsilon E_{x},\\
        &\frac{\partial H_{x}}{\partial z}-\frac{\partial H_{z}}{\partial x}=i k_{0} \epsilon E_{y},\\
        &\frac{\partial H_{y}}{\partial x}-\frac{\partial H_{x}}{\partial y}=i k_{0} \varepsilon E_{z},\\
        &\frac{\partial E_{z}}{\partial y}-\frac{\partial E_{y}}{\partial z}=-i k_{0}\left(\mu H_{x}+i \mu_{a} H_{y}\right),\\
        &\frac{\partial E_{x}}{\partial z}-\frac{\partial E_{z}}{\partial x}=-i k_{0}\left(-i \mu_{a} H_{x}+\mu H_{y}\right),\\
        &\frac{\partial E_{y}}{\partial x}-\frac{\partial E_{x}}{\partial y}=-i k_{0} \mu_{ \|} H_{z}
    \end{aligned}
    \label{eq:2:2}
\end{equation}
Рассмотрим распространение однородной плоской волны вдоль оси z, параллельной вектору $\vH_0$ (случай продольно
намагниченного феррита). В этом случае уравнения \eqref{eq:2:2} принимают вид $\left(\vE(\vr), \vH(\vr) \propto e^{-i h
z}\right)$
\begin{equation}
    \begin{array}
    {cl}{h H_{y}=k_{0} \epsilon E_{x},} & {h E_{y}=-k_{0}\left(\mu H_{x}+i \mu_{a} H_{y}\right)}, \\
    {h H_{x}=-k_{0} \epsilon E_{y},} & {h E_{x}=k_{0}\left(-i \mu_{a} H_{x}+\mu H_{y}\right)}, \\
    {} & {E_{z}=H_{z}=0}
    \end{array}
    \label{eq:2:3}
\end{equation}
Из \eqref{eq:2:3} получаем
\begin{equation}
\begin{array}
    {l}{i k_{0}^{2} \epsilon \mu_{a} E_{x}+\left(h^{2}-k_{0}^{2} \varepsilon \mu\right) E_{y}=0} \\
     {\left(h^{2}-k_{0}^{2} \varepsilon \mu\right) E_{x}-i k_{0}^{2} \varepsilon \mu_{a} E_{y}=0}
    \end{array}
    \label{eq:2:4}
\end{equation}
Система линейных уравнений \eqref{eq:2:4} имеет нетривиальное решение, если ее определитель равен нулю:
\begin{equation}
    (h^2-k_0^2 \epsilon \mu)^2-k_0^4\epsilon^2\mu_a^2=0,
    \label{eq:2:5}
\end{equation}
откуда следует
\begin{equation}
    h_{1,2}^{2}=k_{0}^{2} \varepsilon\left(\mu \pm \mu_{a}\right)
    \label{eq:2:6}
\end{equation}
Таким образом, в продольно намагниченном феррите могут распространяться две поперечные волны с различными постоянными
распространения, т. е. с различными фазовыми скоростями и затуханиями.

Для волны, бегущей в положительном направлении оси z с постоянной распространения $h_{1}=k_{0} n_{1}=k_{0} \sqrt{\epsilon\left(\mu+\mu_{a}\right)}$, из
уравнений \eqref{eq:2:3}-\eqref{eq:2:6} имеем соотношения
\begin{equation}
    E_{x}=i E_{y}, \quad H_{x}=i H_{y}, \quad \frac{E_{x}}{H_{y}}=-\frac{E_{y}}{H_{x}}=\sqrt{\frac{\mu+\mu_{a}}{\varepsilon}}
    \label{eq:2:7}
\end{equation}
а для волны с постоянной распространения $h_{2}=k_{0} n_{2}=k_{0} \sqrt{\varepsilon\left(\mu-\mu_{a}\right)}$ —
соотношения
\begin{equation}
    E_{x}=-i E_{y}, \quad H_{x}=-i H_{y}, \quad \frac{E_{x}}{H_{y}}=-\frac{E_{y}}{H_{x}}=\sqrt{\frac{\mu-\mu_{a}}{\varepsilon}}
    \label{eq:2:8}
\end{equation}
Полученные выражения показывают, что поля $\vE$ и $\vH$ обеих волн взаимно перпендикулярны и поляризованы по кругу с правым
направлением вращения векторов $\vE,\vH$ относительно направления постоянного магнитного поля $\vH_0$ у первой волны ($h_1$) и с
левым — у второй ($h_2$).

Из \eqref{eq:2:3}-\eqref{eq:2:7} следует, что для правополяризованной волны, распространяющейся в положительном направлении оси $z$, справедливы соотношения
\begin{equation}
    \begin{array}
    {ll}{\vE=E_{0}\left(\vx_{0}-i \vy_{0}\right) e^{-i h_{1} z},} & {\vH=\sqrt{\frac{\epsilon}{\mu_{1}}} E_{0}\left(i \vx_{0}+\vy_{0}\right) e^{-i h_{1} z}} \\
    {h_{1}=k_{0} \sqrt{\varepsilon \mu_{1}}=\alpha_{1}-i \beta_{1},} & {v_{\text{ф} 1}=\omega / \alpha_{1}}\end{array}
    \label{eq:2:9}
\end{equation}

где $\mu_{1}=\mu+\mu_{a}=\mu_{1}^{\prime}-i \mu_{1}^{\prime \prime}$ - эквивалентная магнитная проницаемость (см. рис. \ref{fig:3}), $v_{\text{ф}1}$ — фазовая скорость волны.

Магнитная проницаемость $\mu_{1}$ при значениях $\omega_{H}$, близких к круговой частоте волны $\omega$ (то есть при
магнитном поле $\vH_0$, близком к резонансному значению $H_{\text{рез}} = \frac{mc\omega}{e}$ (см. рис. \ref{fig:3})) имеет
большую мнимую часть. Поэтому правополяризованная волна может испытывать значительное резонансное поглощение.

Для левополяризованной волны, распространяющейся в положительном направлении оси $z$, имеем
\begin{equation}
    \begin{array}
    {ll}{\vE=E_{0}\left(\vx_{0}+i \vy_{0}\right) e^{-i h_{2} z},} & {\vH=\sqrt{\frac{\epsilon}{\mu_{2}}} E_{0}\left(-i \vx_{0}+\vy_{0}\right) e^{-i h_{2} z}} \\
    {h_{2}=k_{0} \sqrt{\varepsilon \mu_{2}}=\alpha_{2}-i \beta_{2},} & {v_{\text{ф} 2}=\omega / \alpha_{2}}\end{array}
    \label{eq:2:10}
\end{equation}
где $\mu_{2}=\mu+\mu_{a}=\mu_{2}^{\prime}-i \mu_{2}^{\prime \prime}$.
Эта волна, в отличие от правополяризованной, не испытывает резонансного поглощения (рис. \ref{fig:3}).
\begin{figure}[h!]
    \centering
    \includegraphics[width = 0.6\linewidth]{example-image-a}
    \caption{Зависимости действительных и мнимых частей эквивалентных магнитных проницаемости
    $\mu_{1,2}=\mu_{1,2}^{\prime}-i \mu_{1,2}^{\prime \prime}$ ддя воли с правой и левой круговой поляризацией в
    продольно намагниченном феррите от величины под-магничивающего ноля}
    \label{fig:3}
\end{figure}
Нужно подчеркнуть, что направление вращения (правое или левое) для воли с круговой поляризацией определяется
относительно оси $z$, которая всегда совмещается с направлением постоянного поля подмагничивания, независимо от того, в
каком направлении происходит распространение волны. При распространении в отрицательном направлении оси $z$
левополяризованиая относительно направления распространения волна обладает теми же свойствами, что и правополяризованная
относительно $\vH_0$ волна, распространяющаяся в сторону положительных значений $z$. Аналогичное утверждение справедливо и в
отношении волны с противоположным направлением вращения векторов $\vE,\vH$ (докажите эти утверждения самостоятельно).

При распространении воли в гиротропной среде часто приходится сталкиваться с одновременным присутствием сразу двух волн
— с левой и правой поляризацией. Покажем, что различие фазовых скоростей этих волн приводит к вращению плоскости
поляризации ноля, являющегося их суперпозицией.

Рассмотрим распространение в феррите двух волн с левым и правым направлениями вращения векторов ноля и одинаковыми
амплитудами $E_1 = E_2=E_0/2$. Суммарное поле без учета поглощения имеет вид:
\begin{equation}
\begin{aligned} 
    \vE &=E_{1}\left(\vx_{0}-i \vy_{0}\right) e^{-i \alpha_{1} z}+E_{2}\left(\vx_{0}+i \vy_{0}\right) e^{-i \alpha_{2} z}=\\
    &=\frac{E_{0}}{2} e^{-i \alpha_{0} z}\left[\left(\vx_{0}-i \vy_{0}\right) e^{-i R_{0} z}+\left(\vx_{0}+i \vy_{0}\right) e^{i R_{0} z}\right]=\\
    &=E_{0} e^{-i \alpha_{0} z}\left(\vx_{0} \cos R_{0} z-\vy_{0} \sin R_{0} z\right)
 \end{aligned}
    \label{eq:2:11}
\end{equation} 
где $\alpha_{0}=\left(\alpha_{1}+\alpha_{2}\right) / 2, R_{0}=\left(\alpha_{1}-\alpha_{2}\right) / 2$. Из \eqref{eq:2:11} видно,
что в каждом сечении $z = \const$ поле поляризовано линейно. Однако направление вектора $\vE$ меняется в зависимости от
координаты $z$. Угол $\varphi$ между осью $x$ и плоскостью, проходящей через
вектора $\vE$ и $\vz_0$, определяется, согласно \eqref{eq:2:11}, выражением $\varphi=R_{0} z$. При изменении направления распространения волн на
противоположное вращение вектора $\vE$ относительно $\vH_0$ происходит в прежнем направлении. В этом проявляются невзаимные
свойства продольно намагниченного феррита. Явление вращения плоскости поляризации в магиитоактивпой
среде называют \textit{эффектом Фарадея}, а величину $R_{0}=k_{0}
\sqrt{\varepsilon}(\sqrt{\mu^{\prime}+\mu_{a}^{\prime}}-\sqrt{\mu^{\prime}-\mu_{a}^{\prime}}) / 2$, определяющую
скорость вращения плоскости поляризации, — \textit{постоянной Фарадея}.
% \subsubsection{Поперечное распространение электромагнитных волн в гиромагнитной среде}
% \subsection{Волновод, слабо заполненный ферритом}
% \subsection{Рассчет постоянных распространения собственных мод в волноводе, слабо заполненном ферритом}

\newpage
\section{Экспериментальная часть}

% Схема установки:
% \begin{figure}[H]
%     \centering
%     \includegraphics[width = 0.7\linewidth]{imgs/exp.png}
% \end{figure}
Оборудование: 
\begin{itemize}
    \item Генератор СВЧ излучения с регулируемыми частотой и ослаблением.
    \item Волноводный переключатель с взаимным фазовращателем
    \item Циркулятор на эффекте Фарадея
    \item Волноводный вентиль
    \item Измерительный тракт
    \item Согласованные нагрузки
\end{itemize}
Во всех экспериментах частота генератора $f_g$ была постоянной и не изменялась: $f_g = 10.6$ ГГц.
\subsection{Волноводный переключатель}
Взаимный фазовращатель представляет собой отрезок прямоугольного волновода с продольно намагниченным ферритом,
расположенным вдоль центральной оси волновода. Путем изменения поля подмагничивания фазовращатель
позволяет регулировать набег фазы на участке волновода с ферритом. Два таких фазовращателя используются в волноводном
переключателе, изображенном схематически на рис. \ref{fig:ex:1}. В переключателе мощность высокочастотных электромагнитных колебаний,
поступающих из генератора, делится поровну между двумя волноводными секциями, в которых помещены одинаковые ферритовые
стержни, подмагничиваемые с помощью соленоидов. Размеры секций подобраны таким образом, что в каждой из них (при
используемой частоте генератора) распространяющейся является только волна низшего типа $TE_{10}$. Затем высокочастотная
мощность поступает в волноводно-щелевой мост, па выходе которого перераспределяется между двумя волноводами.

\begin{figure}[h!]
    \centering
    \includegraphics[width = 0.5\linewidth]{example-image-a}
    \caption{Волноводный переключатель с взаимным фазовращателем}
    \label{fig:ex:1}
\end{figure}

Волноводно-щелевой мост сконструирован таким образом, что мощность, поступающая в любое из его плеч (1 или 2), поровну
распределяется между противоположными плечами (3 или 4), причем фаза волны в дальнем плече отстает от фазы волны в
ближнем плече на $\pi/2$. Мост имеет вид сдвоенных прямоугольных волноводов, в общей широкой стенке которых прорезана одна
или несколько щелей. В лабораторной установке щели прорезаны от одной узкой стенки до другой, а размеры $a_1=a$ и $b_1$
поперечного сечения области, образованной сдвоенными волноводами (область связи), выбраны такими, чтобы в пей могли
распространяться волны только трех типов $TE_{10}$, $TE_{01}$, $TE_{11}$.
\begin{figure}[h!]
    \centering
    \includegraphics[width = 0.4\linewidth]{example-image-b}
    \caption{Векторные диаграммы, поясняющие работу волноводно-щелевого моста}
    \label{fig:ex:2}
\end{figure}
Работу волноводно-щелевого моста удобно пояснить с помощью векторных диаграмм (рис. \ref{fig:ex:2}). На таких диаграммах поле в
соответствующем плече моста изображается вектором, длина которого равна абсолютному значению поля, а угол поворота
относительно направления $\phi = 0$ — фазе поля. 

\subsection{Циркулятор на эффекте Фарадея}
Циркулятор на эффекте Фарадея представляет собой отрезок круглого волновода с помещенным внутри него ферритом,
подмагничивание которого осуществляется продольным полем соленоида. К каждому из концов циркулятора подключены два
взаимно-перпендикулярных волновода, причем волноводы, подходящие к противоположным концам циркулятора, повернуты
относительно друг друга па 45° (рис. \ref{fig:exp:circulator}).
\begin{figure}[h!]
    \centering
    \includegraphics[width = 0.5\linewidth]{example-image-a}
    \caption{}
    \label{fig:exp:circulator}
\end{figure}
Циркулятор работает следующим образом. Допустим, что высокочастотная мощность поступает в систему через волновод 1. Это
поле возбуждает в круглом волноводе волну ТЕц соответствующей поляризации. В рассматриваемом случае, как видно из рис.
\ref{fig:exp:circulator}а, поток энергии в волноводе 3 равен пулю и плечо 3 циркулятора не возбуждается. При определенном значении поля
подмагпичивания в феррите плоскость поляризации волны $TE_{11}$ в противоположном конце круглого волновода оказывается
повернутой на 45° относительно своего исходного положения (см. рис. \ref{fig:exp:circulator}б). В этом случае высокочастотная мощность
поступает в волновод 2, а волновод 4 не возбуждается, т. е. мощность поступает из первого плеча циркулятора во второе.

Если при той же величине поля подмагпичивания циркулятор запитывать из второго плеча, энергия не пойдет по обратному
пути. Плоскость поляризации волны $TE_{11}$ будет поворачиваться в том же направлении относительно статического магнитного
поля (в изображенном случае — по часовой стрелке). Такое поле не может возбуждать 1-й волновод, возбуждается лишь 3-й
волновод (рис. \ref{fig:exp:circulator}в).

Аналогичным способом можно показать, что высокочастотная мощность может передаваться из третьего плеча в четвертое, а из
четвертого — в первое. Таким образом, описанный циркулятор работает по схеме
 $1 \rightarrow 2 \rightarrow 3 \rightarrow 4 \rightarrow 1.$

\subsection{Волноводный вентиль}

\subsection{Вывод}

\newpage
\section{Приложение}

\end{document}